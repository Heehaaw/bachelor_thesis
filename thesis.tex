%%% Hlavní soubor. Zde se definují základní parametry a odkazuje se na ostatní části. %%%

%% Verze pro jednostranný tisk:
% Okraje: levý 40mm, pravý 25mm, horní a dolní 25mm
% (ale pozor, LaTeX si sám přidává 1in)
\documentclass[12pt,a4paper]{report}
\setlength\textwidth{145mm}
\setlength\textheight{247mm}
\setlength\oddsidemargin{15mm}
\setlength\evensidemargin{15mm}
\setlength\topmargin{0mm}
\setlength\headsep{0mm}
\setlength\headheight{0mm}
% \openright zařídí, aby následující text začínal na pravé straně knihy
\let\openright=\clearpage

%% Pokud tiskneme oboustranně:
% \documentclass[12pt,a4paper,twoside,openright]{report}
% \setlength\textwidth{145mm}
% \setlength\textheight{247mm}
% \setlength\oddsidemargin{15mm}
% \setlength\evensidemargin{0mm}
% \setlength\topmargin{0mm}
% \setlength\headsep{0mm}
% \setlength\headheight{0mm}
% \let\openright=\cleardoublepage

%% Pokud používáte csLaTeX (doporučeno):
%\usepackage{czech}
%% Pokud nikoliv:
\usepackage[czech]{babel}
\usepackage[T1]{fontenc}

%% Použité kódování znaků: obvykle latin2, cp1250 nebo utf8:
\usepackage[utf8]{inputenc}

%% Ostatní balíčky
\usepackage{graphicx}
\usepackage{amsthm}

%% Balíček hyperref, kterým jdou vyrábět klikací odkazy v PDF,
%% ale hlavně ho používáme k uložení metadat do PDF (včetně obsahu).
%% POZOR, nezapomeňte vyplnit jméno práce a autora.
\usepackage[unicode]{hyperref}   % Musí být za všemi ostatními balíčky
\hypersetup{pdftitle=Vývoj hlasově ovládaných webových her pomocí CloudASR}
\hypersetup{pdfauthor=Jan Milota}

%%% Drobné úpravy stylu

% Tato makra přesvědčují mírně ošklivým trikem LaTeX, aby hlavičky kapitol
% sázel příčetněji a nevynechával nad nimi spoustu místa. Směle ignorujte.
\makeatletter
\def\@makechapterhead#1{
  {\parindent \z@ \raggedright \normalfont
   \Huge\bfseries \thechapter. #1
   \par\nobreak
   \vskip 20\p@
}}
\def\@makeschapterhead#1{
  {\parindent \z@ \raggedright \normalfont
   \Huge\bfseries #1
   \par\nobreak
   \vskip 20\p@
}}
\makeatother

% Toto makro definuje kapitolu, která není očíslovaná, ale je uvedena v obsahu.
\def\chapwithtoc#1{
\chapter*{#1}
\addcontentsline{toc}{chapter}{#1}
}

\begin{document}

% Trochu volnější nastavení dělení slov, než je default.
\lefthyphenmin=2
\righthyphenmin=2

%%% Titulní strana práce

\pagestyle{empty}
\begin{center}

\large

Univerzita Karlova v Praze

\medskip

Matematicko-fyzikální fakulta

\vfill

{\bf\Large BAKALÁŘSKÁ PRÁCE}

\vfill

\centerline{\mbox{\includegraphics[width=60mm]{img/logo.png}}}

\vfill
\vspace{5mm}

{\LARGE Jan Milota}

\vspace{15mm}

% Název práce přesně podle zadání
{\LARGE\bfseries Vývoj hlasově ovládaných webových her pomocí CloudASR}

\vfill

% Název katedry nebo ústavu, kde byla práce oficiálně zadána
% (dle Organizační struktury MFF UK)
Ústav formální a aplikované lingvistiky

\vfill

\begin{tabular}{rl}

Vedoucí bakalářské práce: & Mgr. Ing. Filip Jurčíček, Ph.D. \\
\noalign{\vspace{2mm}}
Studijní program: & Informatika \\
\noalign{\vspace{2mm}}
Studijní obor: & Obecná  Informatika \\
\end{tabular}

\vfill

% Zde doplňte rok
Praha 2015

\end{center}

\newpage

%%% Následuje vevázaný list -- kopie podepsaného "Zadání bakalářské práce".
%%% Toto zadání NENÍ součástí elektronické verze práce, nescanovat.

%%% Na tomto místě mohou být napsána případná poděkování (vedoucímu práce,
%%% konzultantovi, tomu, kdo zapůjčil software, literaturu apod.)

\openright

\noindent
Děkuji panu doktorovi Jurčíčkovi za profesionální vedení této práce.

\newpage

%%% Strana s čestným prohlášením k bakalářské práci

\vglue 0pt plus 1fill

\noindent
Prohlašuji, že jsem tuto bakalářskou práci vypracoval samostatně a výhradně
s~použitím citovaných pramenů, literatury a dalších odborných zdrojů.

\medskip\noindent
Beru na~vědomí, že se na moji práci vztahují práva a povinnosti vyplývající
ze zákona č. 121/2000 Sb., autorského zákona v~platném znění, zejména skutečnost,
že Univerzita Karlova v Praze má právo na~uzavření licenční smlouvy o~užití této
práce jako školního díla podle §60 odst. 1 autorského zákona.

\vspace{10mm}

\hbox{\hbox to 0.5\hsize{%
V ........ dne ............
\hss}\hbox to 0.5\hsize{%
Podpis autora
\hss}}

\vspace{20mm}
\newpage

%%% Povinná informační strana bakalářské práce

\vbox to 0.5\vsize{
\setlength\parindent{0mm}
\setlength\parskip{5mm}

Název práce:
Vývoj hlasově ovládaných webových her pomocí CloudASR
% přesně dle zadání

Autor:
Jan Milota

Katedra:  % Případně Ústav:
Ústav formální a aplikované lingvistiky
% dle Organizační struktury MFF UK

Vedoucí bakalářské práce:
Mgr. Ing. Filip Jurčíček, Ph.D.
% dle Organizační struktury MFF UK, případně plný název pracoviště mimo MFF UK

Abstrakt:
% abstrakt v rozsahu 80-200 slov; nejedná se však o opis zadání bakalářské práce
Cílem práce je navrhnout a vyvinout software pro výuku jazyků hrou za použití webových technologií 
a~čerstvě vznikající CloudASR knihovny. Běžný uživatel provozuje interakci se svým prohlížečem skoro výhradně prostřednictvím
myši a~klávesnice. Díky softwaru v této práci dokumentovanému má nyní uživatel možnost zabřednout do někdy
ne úplně populární výuky jazyka i za pomoci svého hlasu, což nabízí zmíněné výuce netušené možnosti, 
obzvláště stran uživatelské interaktivity. Důraz byl kladen na uživatelskou přívětivost, grafickou fidelitu
a~na kompetitivní aspekt výuky využívajíc Facebookovou integraci a bodové hodnotící žebříčky.

Klíčová slova:
Automatické rozpoznávání řeči, ASR, hry, web, HTML5, Javascript

\vss}\nobreak\vbox to 0.49\vsize{
\setlength\parindent{0mm}
\setlength\parskip{5mm}

Title:
% přesný překlad názvu práce v angličtině
Development of speech enabled web games using CloudASR

Author:
Jan Milota

Department:
Institute of Formal and Applied Linguistics
% dle Organizační struktury MFF UK v angličtině

Supervisor:
Mgr. Ing. Filip Jurčíček, Ph.D.
% dle Organizační struktury MFF UK, případně plný název pracoviště
% mimo MFF UK v angličtině

Abstract:
% abstrakt v rozsahu 80-200 slov v angličtině; nejedná se však o překlad
% zadání bakalářské práce
The main goal of this thesis is to design and implement a piece of software for playful language learning,
using web technologies and the fresh CloudASR library. A common user interacts with their web browser 

Keywords:
% 3 až 5 klíčových slov v angličtině

\vss}

\newpage

%%% Strana s automaticky generovaným obsahem bakalářské práce. U matematických
%%% prací je přípustné, aby seznam tabulek a zkratek, existují-li, byl umístěn
%%% na začátku práce, místo na jejím konci.

\openright
\pagestyle{plain}
\setcounter{page}{1}
\tableofcontents

%%% Jednotlivé kapitoly práce jsou pro přehlednost uloženy v samostatných souborech
\include{intro}
\include{chapter1}
\include{chapter2}

% Ukázka použití některých konstrukcí LateXu (odkomentujte, chcete-li)
\include{example}

\include{summary}

%%% Seznam použité literatury
\include{literature}

%%% Tabulky v bakalářské práci, existují-li.
\chapwithtoc{Seznam tabulek}

%%% Použité zkratky v bakalářské práci, existují-li, včetně jejich vysvětlení.
\chapwithtoc{Seznam použitých zkratek}

%%% Přílohy k bakalářské práci, existují-li (různé dodatky jako výpisy programů,
%%% diagramy apod.). Každá příloha musí být alespoň jednou odkazována z vlastního
%%% textu práce. Přílohy se číslují.
\chapwithtoc{Přílohy}

\openright
\end{document}
