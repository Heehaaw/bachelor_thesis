\chapter*{Závěr}
\addcontentsline{toc}{chapter}{Závěr}

Základním komunikačním kanálem mezilidské interakce zůstává i nadále mluvená řeč.

První polovina práce zvažuje, jak se s tímto faktem v současné době nakládá, jak se k němu přistupuje. Byl zde popsán náhled na mluvené slovo jako na krmivo pro informatické modely, které se mluvu snaží převést do formy stroji srozumitelné --- textu. Byl poskytnut i náhled na novotvar výměny informací, na e-learning, který svým rozmachem dnes tradiční způsoby výuky pomalu překonává.

Hlavním cílem praktické části práce bylo navrhnout a naimplementovat uživatelsky návykovou webovou hru (nebo hry), která bude schopna reagovat na uživatelův hlas. Základním požadavkem bylo vystavět aplikaci na vznikající platformě \verb|CloudASR|. Aplikace měla být zároveň zaměřena na výuku jazyka a na sběr dat pro platformu \verb|CloudASR|.

Druhá polovina práce dokumentuje technologie a myšlenkové a návrhové postupy, jež stály u zrodu aplikace \verb|trogASR|. Tato webová hra zkouší dosáhnout na cíle práce po svém, snaží se být nová a neotřelá. Zároveň však uživatelsky intuitivní a přívětivá. Jedná se o hru jednu, je však nastavitelná širokým spektrem kombinací možností hráči dostupných.

Při vývoji hry a psaní této práce byl kladen extrémní důraz na čistotu zpracování. Byl přizván na pomoc a integrován celý zástup návrhových vzorů a všechna designová rozhodnutí byla činěna s maximální péčí. Psaní kódu bylo prováděno s ohledem na efektivitu, znovupoužitelnost a rozšiřitelnost.

Zmíněných cílů se naplnit podařilo do míry více, než-li uspokojivé. Hra \verb|trogASR| má u uživatelů ohlasy veskrze pozitivní, nejčastějším z nich je zmínka o pěkném grafickém provedení.

Hlavním negativem uživatelské odezvy je fakt, že samotné rozpoznávání řeči jen málokdy vrátí relevantní textové řetězce. Není se však čemu divit. Samotné rozpoznávání řeči je o to kvalitnější, čím větší má nasbíranou datovou základnu. Hra \verb|trogASR| má sloužit právě k obohacování této základny.

Dalšími kroky ve směru, kterým se tato práce vydala, může být jak rozvíjet hru samotnou, tak její koncepty. Zajímavým úkolem by mohlo být do aplikace integrovat nové herní módy, nebo nové statistické algoritmy pro získávání přehledových dat. Zároveň může být dobrým nápadem dobrousit zbylé hrany a doleštit zůstavší skvrnky, které aplikace jistě má.