\chapwithtoc{Úvod}

Základním komunikačním kanálem mezilidské interakce je mluvená řeč. Ačkoli trend v poslední době spěje spíše k separaci této komunikace virtuálními médii, je mluvené slovo i nadále nenahraditelnou součástí socializace a výměny informací. Proč tedy nepřenést schopnost porozumět lidské mluvě i na stroje, jakožto právě na ona separující virtuální média? Současná technologie je k tomu jistě dostačující, výpočetní výkon lidstva roste po zběsilé křivce a technologické know-how po ještě zběsilejší.

Aby tento cíl mohl být dosažen, je kritické mít dobrý \nom{ASR}{Automatic Speech Recognition} (Automatic Speech Recognition) systém. Těchto se pohybuje v éteru mnoho, volně dosažitelných, rozličné kvality a spolehlivosti. Pro specifické využití v této práci byl zvolen systém 
\verb|CloudASR|\footnote{\url{https://github.com/UFAL-DSG/cloud-asr/}}, jehož autorem je kolega Ondřej Klejch.

\texttt{\nom{TrogASR}{Translating Online Game using Automatic Speech Recognition}} (Translating Online Game using Automatic Speech Recognition) je, jak možná název napovídá, online hra, zaměřující se na překlad slov a frází z~jazyka do jazyka za využití automatizovaného rozpoznávání řeči.

Cílem této práce je návrh a implementace graficky přitažlivé hry pro webové prohlížeče, která poskytne uživatelům další, neotřelou, možnost, jak si vyzkoušet a~zdokonalit své jazykové schopnosti. Tato hra pak bude sloužit jako kompetitivitou hnaný motor pro sběr verifikovaných dat pro použitou \verb|CloudASR| knihovnu.

Velký důraz je kladen na grafickou stránku věci; uživatelsky neatraktivní hra nemá valnou naději na šíření mezi uživateli samotnými samovolně. Navíc se na~ošklivou aplikaci nikdo nebude dívat po delší časový úsek, než je jedno sezení, a~už vůbec nikdo se k takovéto aplikaci nebude vracet. Proto bylo veškeré stylování a~malování a~navrhování layoutů a přechodů prováděno s výraznou pílí.

Dále je důraz kladen i na zmíněnou kompetitivitu. Každý má touhu vidět své jméno někde vysoko na žebříčku, zanechávaje své přátele na tomto žebříčku daleko pod sebou. Proto byla zapojena i integrace s Facebookovým API, jež~přináší kýženého výsledku poměrně nenásilnou formou. Na \uv{Like} a \uv{Login} tlačítko jsou uživatelé zvyklí a~nebojí se jej extenzivně používat.

Následující text popisuje a dokumentuje vývoj této aplikace a~myšlenkové pochody při něm stojící. První kapitola analyzuje využití ASR v e-learningu. Druhá kapitola seznámí čtenáře s využitými technologiemi a frameworky pro vývoj aplikace \verb|trogASR|. Kapitola třetí nabídne čtenáři vhled do řešení a~implementace aplikace \verb|trogASR|, do použitých návrhových vzorů a do problematiky s nimi spojené. A konečně, kapitola čtvrtá shrne celou práci.